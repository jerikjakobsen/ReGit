
%====================
% Teaching Assistant
%====================
\subsection{{iOS Teaching Assistant \hfill Jan 2021 --- Nov 2022}}
\subtext{Codepath \hfill Remote, U.S.}
\begin{zitemize}
\item Taught an introduction of programming full stack iOS Applications to a class of 20+ students resulting in a 96\% completion rate
\item Lead breakout room sessions with students to assist with 8 student group projects
\item Provided Q\&A sessions to help students with general questions on programming and iOS Development
\end{zitemize}


%====================
% Software Engineering Intern
%====================
\subsection{{Software Engineer Intern \hfill Jun 2022 --- Aug 2022}}
\subtext{Meta \hfill Menlo Park, California}
\begin{zitemize}
\item Reduced negative messaging interactions on the Instagram iOS app by 14\% by collaborating with design and server teams to implement a feature using Objective-C
\item Improved user experience in iOS and Android by scoping implementation plans for future projects and partnering with cross-functional teams, including creating Objective-C prototypes in close collaboration with the designer
\item Enhanced test coverage of iOS Instagram App modules by 89\% and supported the QA team by developing Objective-C unit tests and implementing end-to-end testing using Jest
\end{zitemize}

%====================
% 
%====================
\subsection{{Facebook University Intern \hfill Jun 2021 --- Aug 2021}}
\subtext{Meta | \href{https://github.com/jerikjakobsen/Pathways}{Github}\hfill Remote, U.S.}
\begin{zitemize}
\item Worked as a full stack iOS application developer, utilizing the MVC design pattern for interface creation and implementation
\item Developed user path visualization using Objective-C and iOS Google Maps SDK, with Parse as the backend, enabling users to track and analyze their routes for improved navigation and planning
\item Streamlined data storage and accelerated load times for users by implementing a custom algorithm that optimized user path storage and performance, eliminating up to 60\% of redundant points for each user path
\end{zitemize}

%====================
% EXPERIENCE E
%====================
%\subsection{{ROLE / PROJECT E \hfill MMM YYYY --- MMM YYYY}}
%\subtext{company E \hfill somewhere, state}
%\begin{zitemize}
%\item In lobortis libero consectetur eros vehicula, vel pellentesque quam fringilla.
%\item Ut malesuada purus at mi placerat dapibus.
%\item Suspendisse finibus massa eu nisi dictum, a imperdiet tellus convallis.
%\item Nam feugiat erat vestibulum lacus feugiat, efficitur gravida nunc imperdiet.
%\item Morbi porta lacus vitae augue luctus, a rhoncus est sagittis.
%\end{zitemize}
